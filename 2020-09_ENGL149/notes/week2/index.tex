\chapter{Week 2}

\section{9/21 -- Hour 5}

\subsection{Writing Prompt}

\begin{enumerate}
    \item Primary: HR person or whoever decides you get interviewed. Secondary: your potential future coworkers (engineering department), whoever interviews you
    \item To filter out applicants in the first stage
    \item Software scanning. The HR person is in a work from home environment. Possibly more or less formal depending on industry, consider word choice. Geographic context.
\end{enumerate}

\subsection{Project 1}

\begin{itemize}
    \item Create a spreadsheet for tracking job applicants
    \item NOTE: ask professor if we could review my website
    \item Tailor resume for each job -- software intern is probably uniform enough
    \item Look at cal poly guide
    \item Look into the resume brainstorm. Work: \#hours/week, quantitative stuff. Projects: look at CPE 133/233? 
    \item Leadership? Do I even have any? Ask professor about somewhat radical activism.
    \item Place IEEE, SWE on the list. Also general ham radio on website (FCC part 97).
    \item Hobbies and interests? Cooking, erhu, 3D printing
    \item Qualitative aspects. Likely not on resume, but to keep in mind.
    \item Discouraged objective line
    \item Education: cal poly. Maybe include community college during HS
    \item Experience: Ask professor about quotes. ``Software engineer @ Apple'' instead of ``Apple -- Software engineer.'' Quantifiable metrics.
    \item Projects: astrid.tech and collision zone?
    \item Extra categories: use software skills
\end{itemize}

\section{Hour 8}

\begin{outline}
    \1 Caution when adapting sample letters and previous letters 
    \1 Active voice, not passive. ``I did this'' instead of ``this was done to me''
    \1 Avoid ``various'' and be specific
    \1 Avoid being overly formal or informal
    \1 Polish, must be perfect
    \1 Forbes 
        \2 Always write a cover letter, unless they explicitly say no. Even if they don't ask for one.
        \2 1/2 to 2/3 of the page 
        \2 Not in the resume: story, why want, what can bring 
        \2 Good opener (``excited about things I could bring to X'')
        \2 Name dropping 
        \2 Anecdotes
        \2 Do not copy/paste job description, resume
    \1 Cover Letter Strategies 
        \2 Make it personal -- find the person's name. Call them.
        \2 Begin with why them? Don't begin with why me. Cite specific reasons. Look at their news page. Look at their founding story. Something genuine.
        \2 Next, why you. Tell them how I would improve their business.
        \2 Show passion -- Why I want to do it. ``I would love the opportunity to\dots'' Exaggerate. Dream job. Even if not.
        \2 Show kindness -- think about the person reading it. 
\end{outline}

\begin{verbatim}
Astrid Yu
1515 Arc Way Apt 307
Burlingame, CA 94010
(650) 483-0527
astrid@astrid.tech

September 30, 2020

HR Representative
Amazon, Inc.
410 Terry Ave. North
Seattle, WA
98109-5210

Subj: Software Developer Intern

Dear Amazon HR Representative:
\end{verbatim}

\subsection{Why them paragraph}

\begin{outline}
    \1 Mention specifics. Be more specific about why you like them, not just the company website's buzzwords.
        \2 Personal connection: using their products
        \2 Toured the company 
        \2 Know someone who worked there 
        \2 Their robot did something funny
        \2 Just find some way in for the connection 
    \1 
\end{outline}
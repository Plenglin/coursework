\chapter{Week 1}

\section{9/14 -- Hour 1}

\subsection{Writing Prompt: Define "technical communication" in your own words.}

Technical means it's precise and descriptive, and communication means it's a way of getting ideas across. So, technical communication is essentially a broad term that describes various kinds of techniques used to communicate an idea in a field that requires it.
  
\subsection{Weekly Deliverables}

Everything due by 9pm Friday

\begin{itemize}
    \item Discussion posts and replies
    \item Assignments via PDF uploads
    \item Quizzes
\end{itemize}

Also, 3 projects.

\subsection{Technical communication}

\begin{outline}
    \1 Exchange of information to accomplish some kind of technical goal
\end{outline}

\subsection{Nick Belling's video resume}

\begin{itemize}
    \item Entertaining and funny
    \item Unique -- not many people use video resume
    \item Special effects
    \item Names the specific things he's done and what technology was used
    \item Call to action at the end
\end{itemize}

Notes are writer-centric. We need to write in a reader-centric way.

\begin{itemize}
    \item Headings help organize
    \item Indent B-level headings
    \item Don't use dash as bullet point because of possible minus/negative impression
\end{itemize}

\subsection{Audience, Purpose, Context}

Results of technical writers failing?

\begin{itemize}
    \item Death
    \item Legal issues 
    \item Sexual harrassment due to tone confusion
    \item Financial cost
\end{itemize}

\section{9/14 -- Hour 2}

\subsection{Writing prompt: ``The Importance of Writing in Tech-Related Fields.''}

\subsubsection{How convinced are you that writing is as important as its proponents are saying?}

I'm fairly convinced that writing is important from my own experience. My experience programming, both alone and on a team, has proven that to me. I've read tons and tons of poorly-documented libraries and code and I've had to piece together what it really means myself, and that's annoying. 

At my internships, it's even more crucial. I've written lots and lots of reports of different sizes describing what I've done, what went wrong, etc.

\subsubsection{What do you honestly believe you need to have or learn (i.e. what specific writing skills), if any, to succeed in this class and beyond as an engineer?}

I feel like I could be a bit more concise with my language. In addition, I often ramble a bit too much, and my ideas can sometimes be a bit loosely organized.

\begin{outline}
    \1 Focus on the reader, not the writer 
        \2 Think about children's picture books
    \1 Document must be efficient and accessible
        \2 No extraneous information
        \2 Link to extra information
    \1 Clear and relevant writing
\end{outline}

\subsubsection{Primary purposes of technical communication}

\begin{outline}
    \1 Informational
        \2 Objective, static, convey information. 
        \2 No subjectivity
        \2 Example: Report cards
        \2 Answer ``who,'' ``what,'' ``when,'' ``where''
    \1 Instructional    
        \2 How to enroll in a class 
        \2 Typically in steps
        \2 Dynamic
        \2 Answer ``how''
    \1 Persuasive
        \2 Subjectivity
        \2 All documents are at least in part persuasive
        \2 Answer ``why''
\end{outline}

\subsubsection{Additional concepts}

Technical communication is\dots

\begin{outline}
    \1 Global
        \2 $\$41 = 41,000$ KRW
        \2 Consider ettiquite, avoiding offending others. Good ideas can get shrouded in local customs
    \1 Collaborative
        \2 Good group work happens when people communicate well
    \1 Digital
        \2 Some places, paper is necessary
        \2 Otherwise, PDFs are standard
\end{outline}

Audience, purpose, context

\section{9/16 -- Hour 3}

\subsection{Usable documents}

\begin{outline}
    \1 Locate the information they need easily
    \1 Understand the information immediately
    \1 Use the information safely and successfully
        \2 Safety warnings at the right location
\end{outline}

Who matters most? The audience. We use audience analysis

\subsection{Audience Analysis}

\begin{outline}
    \1 Who is the main audience for this document? (primary audience)
    \1 Who else is likely to read it? (secondary audiences)
    \1 What is the writer's relationship with the audience? Are there multiple types of relationships involved? (to be further discussed below; power connection, relationship connection, rational connection, something else?)
        \2 Build a relationship with them
        \2 Leverage existing relationships
    \1 How familiar might the audience be with technical details? (demographics, features, skills, needs, etc.)
    \1 What culture or cultures does your audience represent? (demographics, features, skills, needs, etc.)
\end{outline}

\subsection{Assignment 1}

Take the technical instructions and do an analysis of it (\href{https://canvas.calpoly.edu/courses/26844/pages/class-1-dot-2-w-9-slash-16-hours-3-4?module_item_id=695646}{instructions})

\begin{outline}
    \1 Audience analysis 
        \2 Secondary audiences
            \3 the others who might be involved in this besides the user
            \3 Legal departments, safety 
        \2 Characteristics 
            \3 Possible stereotyping, but ah well
        \2 Skillset
    \1 Purpose analysis, should be instructional
    \1 Context analysis 
        \2 What environment is it in? What external factors for the product?
        \2 Big picture: current events? Season of the year?
        \2 Scrutinized (i.e. Barbie)?
        \2 Usability (i.e. e-books need clickability/hyperlinking)
\end{outline}

\subsection{Types of Audience Connections}

\begin{outline}
    \1 Power 
        \2 ``Because I said so''
        \2 Boss/employee, parent/child 
        \2 Not nice for low-stakes stuff, rarely good
        \2 First-responder connections are better for this (``get out immediately'')
    \1 Relationship 
        \2 ``Because I'd appreciate it'
        \2 Family, friend, spouse, cross-cultural
        \2 Favors
    \1 Rational 
        \2 ``Because it makes sense''
        \2 Usually the best approach
\end{outline}

\section{9/16 -- Hour 4}

\begin{enumerate}
    \item Primary: The students. \st{Secondary: employers, people related to students?} Secondary: Prof. Green, possibly graders, his colleagues, his boss if he had one
    \item Informational, some instructional
    \item Remote classroom context, over Zoom, required, low-morale engineering students
    \item Rational most of the time 
    \item Mostly usable with bolded headings and such. Could be better with larger headlines. Async class: it's usable because it's there all the time
\end{enumerate}

\subsection{Consider your context}

\begin{outline}
    \1 Formal, semi-formal, informal 
    \1 Based on the audience 
    \1 What's going on in the world?
        \2 BLM 
        \2 \#MeToo (no longer used)
        \2 People's gender (professor says trans rights)
\end{outline}

\subsection{CDC}
\begin{outline}
    \1 Different languages 
    \1 Suggestion: dynamic visual
    \1 Bullets not good for instructions. Numbers are.
\end{outline}

\subsection{Final thoughts}

\begin{outline}
    \1 Context
        \2 Length should be approrpiate 
        \2 Budget for the user and for the printer
    \1 Purpose statement 
        \2 Write for high-stakes things 
\end{outline}


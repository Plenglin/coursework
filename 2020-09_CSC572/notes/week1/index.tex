\chapter{Week 1}

\section{2020-09-15}

\subsection{Introduction}

\subsubsection{Every week}

\begin{itemize}
    \item Videos
    \item Meet Tuesday to discuss videos (synchronous part)
    \item Thursday is extended office hours
    \item Lab due Friday
    \item No midterms, but presentations
\end{itemize}

\subsubsection{Presentations}

\begin{itemize}
    \item Present in groups of 3 or 4, shuffled
\end{itemize}

\subsubsection{Environments to set up}

\begin{itemize}
    \item OpenGL Libraries
    \item Compute shaders not allowed on Mac?
    \item Windows - Use Visual Studio
    \item Linux - Ask Sydney
\end{itemize}

\subsection{Lecture}

\subsubsection{3D Graphics}

\begin{outline}
    \1 How to store 3D stuff?
        \2 Things are completely hollow
        \2 Surface is triangles. 3 points = a plane. 4 points = saddle
        \2 The 3D modeler has given you the mesh already
        \2 Easiest case: float triplets make a point, point triplets make a triangle
    \1 Rendering a triangle?
        \2 Screen has a fixed number of discrete pixels on a 2D screen
        \2 2 coordinate systems: world (3D float) and screen (2D int). 
        Matrix $\mathbf{M}$ such that $\mathbf{M}\vec{v_w} = \vec{v_s}$
        \2 Triangle needs to get filled in
    \1 Computation
        \2 First 3D graphics were done on single-core CPUs and were slow and primitive. 
        \2 Originally, CPU writes directly into VRAM that gets directly drawn to monitor, you see the image get drawn in front of you
        \2 Double-buffering: 2 VRAM buffers, write to A, copy A to B when finished, render B
        \2 Page-flipping: 2 VRAM buffers, write to A, render A and start writing to B, etc.
        \2 The VRAM alone is the original "graphics card"
        \2 Now, 150-250k triangles, offload them onto a highly parallel non-CPU processor (GPU), lots of tiny cores, bigger memory, store triangles and pictures on it
        \2 VRAM closer to GPU = higher clock rates
    \1 Shaders 
        \2 Copy data from RAM into VRAM
        \2 Write a shader to work on the memory (named so because it's often used for light calculations)
        \2 Scientists and engineers write shaders for matrices. The course is graphics and simulations

\end{outline}

\section{Video: Render Pipeline}

\subsection{Render pipeline}

Conceptually:

\begin{enumerate}
    \item Transform the original object
    \item Convert 3D float to 2D int
    \item Project to 2D
\end{enumerate}

On the GPU, this is:

\begin{outline}[enumerate]
    \1 Transform vertices (vertex processing)
        \2 Take vertices (in triangles, sorted)
        \2 Translated however you want
    \1 Rasterization (automatic)
        \2 Becomes 2D coordinates
    \1 Pixel (fragment) processing
        \2 For each triangle
\end{outline}

\subsection{How to the CPU tells the GPU to do it}

\begin{outline}[enumerate]
    \1 Initialize OpenGL
    \1 Copy resources (models and textures) to VRAM
        \2 Hard drive $\rightarrow$ CPU $\rightarrow$ VRAM $\rightarrow$ GPU
    \1 Loop for every frame
        \2 Clear background
        \2 Select objects to display
        \2 Activate vertex processing*
        \2 Activate pixel (fragment) processing*
        \2 Render it
\end{outline}

*shaders, written in GLSL

The GPU cores are shader \textbf{units}. The programs are called \textbf{shaders}. They are executed on all cores, and tasks are automatically assigned.

\section{Video: Windowed OS Programming}

\begin{outline}
    \1 Initialize 
    \1 Loop
        \2 Render/Update
            \3 Try not to trap the program counter with a long loop! Put in another thread, or break up the function
        \2 Event handling\footnote{These events come from some other process, and this process gets suspended due to preemptive multitasking}
\end{outline}

\section{Video: Program Structure}

Most files are there for helpers (i.e. window manager)

Functions of note:

\begin{outline}
    \1 \texttt{main}
    \1 \texttt{Application::mouseCallback}, \texttt{Application::keyCallback}
    \1 \texttt{Application::resizeCallback}
\end{outline}

OpenGL does all the communication with the GPU for us. Vulkan is \textit{supposed} to succeed it, Metal is for Apple

\section{Video: Vertex Arrays and Buffers}

\begin{outline}
    \1 In OpenGL, we treat things on a scale of $[-1, 1]$
    \1 Buffers
        \2 Vertex Array Object (VAO)
            \3 Our model, the idea of the object
        \2 Vertex Buffer Object (VBO)
            \2 Contained inside VAOs
            \2 One VBO for positions, another for colors, another for texture coordinates, another for lighting normals...
            \2 Same index for same vertex
    \1 How to work with VRAM?
        \2 \texttt{glGenVertexArrays} -- You get an int reference to a VAO. Like a pointer, except it's just a number, so it's more like a file descriptor
        \2  \texttt{glBindVertexArray} -- Everything under this line works with this VAO now. \texttt{glBindVertexArray(0)} unbinds everything
        \2 \texttt{glGenBuffers} -- same, but for VBOs
        \2 \texttt{glBufferData} -- Copies memory from RAM to VBO
\end{outline}

\section{Video: Render function}

\begin{outline}[enumerate]
    \1 Get buffer size
    \1 Clear display
    \1 \texttt{glBindVertexArray} then \texttt{glDrawArrays}
\end{outline}

\section{Video: Shader Programming}

\begin{outline}[]
    \1 VAO now has triangles in it
    \1 \texttt{layout(location = 0) in vec3 vertPos;} -- take in some data from the VBO
    \1 \texttt{out vec3 color;} -- define the buffer location
    \1 Colors are \texttt{vec3},  just like positions
        \2 vec3.r = vec3.x = the very first element.
    \1 You can pass data between shaders by giving them the same name?
\end{outline}

\section{Video: VBO Locations}


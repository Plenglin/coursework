\documentclass{article}

\usepackage{geometry}
\usepackage{afterpage}
\usepackage{graphicx}
\usepackage{outlines}
\usepackage{hyperref}
\usepackage{changepage}
\usepackage{amssymb}
\usepackage{amsfonts}
\usepackage{pdflscape}
\usepackage{amsmath}
\usepackage{amsthm}
\usepackage{blindtext}

\author{Astrid Augusta Yu}
\title{CSC 348 -- Homework \#5}
\date{\today}

\newcommand{\contradiction}{$\rightarrow\leftarrow$}
\renewcommand\qedsymbol{\texttt{d(\textbf{\^{}\_\^{}})>}}
\newtheorem{theorem}{Theorem}
\newtheorem{lemma}{Lemma}
\newtheorem{claim}{Claim}

\begin{document}
\maketitle
\tableofcontents

\section{Questions}
\begin{outline}[enumerate]
    \1 $f(n) = n^2 + 1$ is not 1-1.
        \begin{proof}
            Consider $-1, 1 \in \mathbb{R}$. Note that $1 \neq -1$, but that $f(-1) = f(1) = 1^2 + 1 = 2$. Thus, $f$ is not 1-1 by definition of 1-1.
        \end{proof}
    \1 $f(x) = n^5$ is 1-1.
        \begin{proof}
            Suppose for some $x, y \in \mathbb{R}$, $f(x) = f(y)$.
            \begin{equation}
                \begin{aligned}
                    x^5 &= y^5 \\
                    \sqrt[5]{x^5} &= \sqrt[5]{y^5} \\
                    x &= y
                \end{aligned}
            \end{equation}

            Thus, by definition of 1-1, $f$ is 1-1.
        \end{proof}
    \1 $f(n) = n - 1$ is onto.
        \begin{proof}
            Let $y \in \mathbb{R}$. Note that $y = (y + 1) - 1 = f(y + 1)$. Since $(y + 1) - 1 \in \mathbb{R}$, $f(x)$ is onto by definition of onto.
        \end{proof}
    \1 $f(n) = n^2 + 1$ is not onto.
        \begin{proof}
            Consider $0 \in \mathbb{R}$. Note that $x = \sqrt{0 - 1} \notin \mathbb{R}$. Because of this, no $x \in \mathbb{R}$ such that $f(x) = x^2 + 1 = 0$ is possible. Therefore, by definition of onto, $f$ is not onto.
        \end{proof}
    \1 $f(n) = \sqrt[5]{n}$ is onto.
        \begin{proof}
            Let $y \in \mathbb{R}$. Note that $y = \left(\sqrt[5]{y}\right)^5 = f\left(\sqrt[5]{y}\right)$. Since $(\sqrt[5]{y})^5 \in \mathbb{R}$, $f(x)$ is onto by definition of onto.
        \end{proof}
    \1 $f: \mathbb{N} \rightarrow \mathbb{N}; f(n) = 2n$ is injective, but not surjective.
    \begin{proof}[Proof of injectivity.]
        Suppose $n = 3$
    \end{proof}
    \begin{proof}[Proof of non-surjectivity.]
        Suppose 
    \end{proof}

    \1 $f: \mathbb{N} \rightarrow \mathbb{N}; f(n) = \sqrt{n}$ is surjective, but not injective.
    \begin{proof}[Proof of non-injectivity.]
        Suppose 
    \end{proof}
    \begin{proof}[Proof of surjectivity.]
        Suppose 
    \end{proof}

    \1 $f: \mathbb{N} \rightarrow \mathbb{N}; f(n) = (-1)^n + n$ is one-to-one and onto.

    \begin{proof}[Proof of injectivity.]
        Suppose $a, b \in \mathbb{N}$ and $f(a) = f(b)$.

        By the definition of those functions,
        \begin{equation}
            (-1)^a + a = (-1)^b + b
        \end{equation}

        Note that the terms $(-1)^a$ and $(-1)^b$ will either be 1 or $-1$, which are both odd. Thus, $(-1)^a$ and $(-1)^b$ are always odd.

        \textbf{Case 1.} $f(a) = f(b)$ is even.
        \begin{adjustwidth}{0.5cm}{0.5cm}
            By Lemma \ref{thm:evenlemma}, since this value is even, its addends have the same sign. Thus, $a$ and $b$ must be odd.
            
            By definition of odd, there exists some $i, j \in \mathbb{Z}$ such that $a = 2i + 1$ and $b = 2j + 1$. 
            \begin{equation}
                \begin{aligned}
                    (-1)^{2i + 1} + a &= (-1)^{2j + 1} + b \\
                    ((-1)^2)^i\cdot (-1)^1 + a &= 
                    ((-1)^2)^j\cdot (-1)^1 + b \\
                    -1 + a &= -1 + b \\
                    a &= b \\
                \end{aligned}
            \end{equation}
        
            Thus, $f(n)$ is 1-1 for even $n$.
        \end{adjustwidth}

        \textbf{Case 2.} $f(a) = f(b)$ is odd.
        \begin{adjustwidth}{0.5cm}{0.5cm}
            By Lemma \ref{thm:evenlemma}, since this value is even, its addends have the same sign. Thus, $a$ and $b$ must be odd.
            
            By definition of odd, there exists some $i, j \in \mathbb{Z}$ such that $a = 2i$ and $b = 2j$. 
            \begin{equation}
                \begin{aligned}
                    (-1)^{2i} + a &= (-1)^{2j} + b \\
                    ((-1)^2)^i + a &= 
                    ((-1)^2)^j + b \\
                    1 + a &= 1 + b \\
                    a &= b \\
                \end{aligned}
            \end{equation}
        
            Thus, $f(n)$ is 1-1 for even $n$.
        \end{adjustwidth}

        Since both cases are true, $f(n)$ is a 1-1 function.
    \end{proof}
    \begin{proof}[Proof of surjectivity.]
        Suppose $y \in \mathbb{N}$. We will prove that there always exists some $n \in \mathbb{N}$ such that $f(n) = y$.

        \textbf{Case 1.} $y$ is even.
        \begin{adjustwidth}{0.5cm}{0.5cm}

            By definition of even, for some $i \in \mathbb{N}$, $y = 2i$.

            Consider $f(y + 1)$.
            \begin{equation}
                \begin{aligned}
                    f(y + 1) &= (-1)^{y + 1} + (y + 1) \\
                    &= (-1)^{2i + 1} + y + 1 \\
                    &= ((-1)^2)^i \cdot (-1)^{1} + y + 1 \\
                    &= 1\cdot(-1) + y + 1 \\
                    &= y
                \end{aligned}
            \end{equation}

            Thus, if $y$ is even, then $f(y + 1) = y$.
        \end{adjustwidth}

        \textbf{Case 2.} $y$ is odd.
        \begin{adjustwidth}{0.5cm}{0.5cm}

            By definition of odd, for some $i \in \mathbb{N}$, $y = 2i + 1$.

            Consider $f(y - 1)$.
            \begin{equation}
                \begin{aligned}
                    f(y - 1) &= (-1)^{y - 1} + (y - 1) \\
                    &= (-1)^{(2i + 1) - 1} + y - 1 \\
                    &= (-1)^{2i} + y - 1 \\
                    &= ((-1)^2)^i + y - 1 \\
                    &= 1 + y - 1 \\
                    &= y
                \end{aligned}
            \end{equation}

            Thus, if $y$ is odd, then $f(y - 1) = y$.
        \end{adjustwidth}

        Therefore, for all $y \in \mathbb{N}$, there exists a value $n \in \mathbb{N}$ such that $f(n) = y$. Thus, $f$ is onto.
    \end{proof}

    \1 
        \2 $(f\circ g)(x) = (e^x)^2 + 1 = e^{2x} + 1$
        \2 $(g\circ f)(x) = e^{x^2 + 1}$
    \1 \begin{theorem}
        Let $f: B \rightarrow C$ and $g: A \rightarrow B$. If $f$ and $g$ are onto, then $f \circ g$ is onto.
    \end{theorem}
    \begin{proof}
        Suppose $c \in C$. By definition of $f$ as an onto function, there exists a $b \in B$ such that $f(b) = c$. It follows that there exists an $a \in A$ such that $g(a) = b$ by definition of $g$ as an onto function. Therefore, there always exists some $a \in A$ such that $(f \circ g)(a) = c$.

        Thus, by definition of onto, if $f$ and $g$ are onto, then $f \circ g$ is onto.
    \end{proof}
    \1
        \2 \begin{theorem}
            Let $f: B \rightarrow C$ and $g: A \rightarrow B$. $f \circ g$ being onto does not imply that $g$ is onto.
        \end{theorem}
        \begin{proof}
            Consider the case when $A = B = C = \mathbb{R}$, $f(x) = \ln x$, and $g(x) = x^2$. It follows that $(f \circ g)(x) = \ln(x^2)$.

            \begin{claim}
                $f$ is onto.
                \begin{adjustwidth}{0.5cm}{0.5cm}
                    \begin{proof}
                        Let $y \in \mathbb{R}$. 
                        Note that $y = \ln \left(e^y\right) = f(e^y)$. 
                        Since $e^y \in \mathbb{R}$, $f(x)$ is onto by definition of onto.  
                    \end{proof}
                \end{adjustwidth}
            \end{claim}

            \begin{claim}
                $g$ is not onto.
                \begin{adjustwidth}{0.5cm}{0.5cm}
                    \begin{proof}
                        Consider $-1 \in \mathbb{R}$. 
                        Note that $x = \sqrt{-1} \notin \mathbb{R}$. 
                        Therefore, there is no such $x \in \mathbb{R}$ such that $g(x) = -1$. 
                        Thus, $g$ is not onto.
                    \end{proof}
                \end{adjustwidth}
            \end{claim}

            \begin{claim}
                $f \circ g$ is onto.
                \begin{adjustwidth}{0.5cm}{0.5cm}
                    \begin{proof}
                        Let $y \in \mathbb{R}$. 
                        Note that $y = \ln \left(\right(e^\frac{y}{2}\left)^2\right) 
                        = f \left( e^\frac{y}{2}\right)$.
                        Since $e^\frac{y}{2} \in \mathbb{R}$, $f(x)$ is onto by definition of onto.  
                    \end{proof}
                \end{adjustwidth}
            \end{claim}

            Notice that $f \circ g$ is onto, but $g$ is not onto. Therefore,  $f \circ g$ being onto does not imply that $g$ is onto.
        \end{proof}
        \2 \begin{theorem}
            Let $f: B \rightarrow C$ and $g: A \rightarrow B$. $f \circ g$ being onto implies that $f$ is onto.
        \end{theorem}
        \begin{proof}
            Suppose $c \in C$. By definition of $f \circ g$ as an onto function, there exists an $a \in A$ such that $(f \circ g)(a) = f(g(a)) = c$. Note that the existence of $f(g(a))$ requires $g(a) \in B$, so $g(a) \in B$.
            
            Let $b = g(a)$. Thus, $f(b) = c$. Therefore, if $f \circ g$ is onto then $f$ is onto.
        \end{proof}
\end{outline} 

\section{Additional Lemmas with Proofs}

\begin{lemma}
    \label{thm:evenlemma}
    Let $a, b \in \mathbb{Z}$. $a + b$ is even if and only if $a$ and $b$ have the same parity, and odd if and only if $a$ and $b$ have different parities. (Reproduced from homework 4).
\end{lemma}
    
\begin{proof}
    
    \textbf{Case 1.} $a$ and $b$ are both even.

    By definition of even:
    \begin{equation}
        \begin{aligned}
            a &= 2k \\ 
            b &= 2l
        \end{aligned}
    \end{equation}

    Therefore,
    \begin{equation}
        \begin{aligned}
            a + b &= 2k + 2l \\
            &= 2(k + l) 
        \end{aligned}
    \end{equation}

    Let $m = k + l$. Then
    \begin{equation}
        a + b = 2m
    \end{equation}

    By definition of even, $a + b$ is even.  
   
    \textbf{Case 2.} $a$ and $b$ are both odd.

    By definition of odd:
    \begin{equation}
        \begin{aligned}
            a &= 2k + 1 \\
            b &= 2l + 1
        \end{aligned}
    \end{equation}

    Therefore, 
    \begin{equation}
        \begin{aligned}
            a + b &= 2k + 1 + 2l + 1 \\ 
            &= 2(k + l + 1) 
        \end{aligned}
    \end{equation}

    Let $m = k + l + 1$. Then,
    \begin{equation}
        a + b = 2m
    \end{equation}

    By definition of even, a + b is even.

    \textbf{Case 3.} $a$ and $b$ have different parity. 
    
    WLOG let $a$ be even and $b$ be odd. By definition of even:
    \begin{equation}
        a = 2k
    \end{equation}

    By definition of odd:
    \begin{equation}
        b = 2l + 1
    \end{equation}

    Therefore,
    \begin{equation}
        \begin{aligned}
            a + b &= 2k + 2l + 1 \\
            &= 2(k + l) + 1
        \end{aligned}
    \end{equation}

    Let $m = k + l$. Then,
    \begin{equation}
        a + b = 2m + 1
    \end{equation}

    By definition of odd, a + b is odd.

    \textbf{In Summary:}

    \begin{center}
        \begin{tabular}{c c | c}
            $a$ parity & $b$ parity & $a + b$ parity \\
            \hline
            even & even & even \\ 
            even & odd & odd \\ 
            odd & even & odd \\ 
            odd & odd & even
        \end{tabular}
    \end{center}

    Therefore, $a + b$ is only even when a's parity $=$ b's parity, and only odd when 
    a's parity $\neq$ b's parity, concluding the lemma.
\end{proof}


\newgeometry{margin=2cm}
\begin{landscape}
    \section{Scratch Work}

    \centering
    \thispagestyle{empty}
%    \includegraphics[width=0.95\linewidth]{1.jpg}

\end{landscape}

\restoregeometry
    
\end{document}
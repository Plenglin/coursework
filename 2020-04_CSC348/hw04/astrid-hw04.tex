\documentclass{article}

\usepackage{geometry}
\usepackage{afterpage}
\usepackage{graphicx}
\usepackage{outlines}
\usepackage{hyperref}
\usepackage{amssymb}
\usepackage{amsfonts}
\usepackage{pdflscape}
\usepackage{amsmath}
\usepackage{amsthm}
\usepackage{blindtext}

\author{Astrid Augusta Yu}
\title{CSC 348 -- Homework \#2}
\date{\today}

\newcommand{\contradiction}{\texttt{;(}}
\renewcommand\qedsymbol{\texttt{d(\textbf{\^{}\_\^{}})>}}
\newtheorem{theorem}{Theorem}
\newtheorem{lemma}{Lemma}
\newtheorem{case}{Case}
\newtheorem{subcase}{Case}
\numberwithin{subcase}{case}

\begin{document}
\maketitle
\tableofcontents

\section{Questions}
\begin{outline}[enumerate]
    \1 
        \2 $(v_3, v_0, v_1)$
        \2 $(v_0, v_2, v_3, v_1, v_2, v_0)$
        \2 $(v_0, v_1, v_2, v_3, v_0)$
        \2 No. A path requires all vertices be unique except for the first 
            and last in the case of a cycle. A cycle of length 5 would require that 
            5 vertices exist, and a path of length 5 with strictly unique first 
            and last vertices would require that 6 vertices exist. Since the graph 
            only has 4 vertices, neither of these are possible.
    \1 
        \2 The highest degree vertex is $v_2$ with degree $4$.
        \2 $(v_0, v_1, v_2, v_2, v_4, v_3, v_0)$
        \2 Yes: $(v_0, v_1, v_2, v_4, v_3, v_0)$
    \1 \begin{theorem}
            Let $G = (V, E)$ be a graph. If $G$ contains an Eulerian circuit $C$, then 
            $C$ passes through every vertex in $G$.
        \end{theorem}
        \begin{proof}
            Suppose, by way of contradiction, that $v \in V$ is a vertex that $C$ does not pass through. 

            By definition of a connected graph, there exists a $u \in V$ such that $(u, v) \in E$.

            Additionally, by definition of Eulerian circuit, $C$ passes through the edge $(u, v)$.

            However, this means $C$ passes through $v$. \contradiction

            Thus, by way of contradiction, $C$ passes through all vertices of $G$.

        \end{proof}
    \1 \begin{theorem}
            Let $m, n \in \mathbb{Z}$. If $nm$ is even, then $n$ or $m$ is even.
        \end{theorem}
        \begin{proof}
            Suppose, by contraposition, that $n$ is odd and $m$ is odd. Therefore, there exists 
            a $i, j \in \mathbb{Z}$ such that:
            \begin{equation}
                \begin{aligned}
                    n &= 2i + 1 \\
                    m &= 2j + 1
                \end{aligned}
            \end{equation}

            It follows that 
            \begin{equation}
                \begin{aligned}
                    nm &= (2i + 1)(2j+1) \\
                    &= 4ij + 2i + 2j + 1 \\
                    &= 2(2ij + i + j) + 1
                \end{aligned}
            \end{equation}

            Suppose $k = 2ij + i + j$. Therefore,
            \begin{equation}
                nm = 2k + 1
            \end{equation}

            Therefore, $nm$ is odd.

            Thus, by contraposition, $nm$ is even if $n$ is even or $m$ is even.
        \end{proof}
    \1 \begin{theorem}
            Let $p, n \in \mathbb{Z}$. If $p$ is prime, then $\gcd(p, n) = 1$ or $\gcd(p, n) = p$.
        \end{theorem}

        \2 \begin{proof}
                Let $g = gcd(p, n)$. By contraposition, suppose that the following are true:
                \begin{equation}
                    \begin{aligned}
                        g &\neq 1 \\
                        g &\neq p                        
                    \end{aligned}
                    \label{eqn:primecontraposassumption}
                \end{equation}

                By definition of gcd, it is known that $g | p$, and by definition of divides, for some $i \in \mathbb{Z}$:
                \begin{equation}
                    p = ig
                    \label{eqn:pig1}
                \end{equation}

                Since $g$ is a factor of $p$, and $g \neq 1 \wedge g \neq p$ (see \ref{eqn:primecontraposassumption}), that means $p$ must be 
                a composite number (not prime).
                
                Therefore, if there exists an integer $n$ such that $gcd(p, n)$ is neither $1$ nor $p$, then $p$ is not prime.

                Thus, by proof by contraposition, if $p$ is prime, then $\gcd(p, n) = 1$ or $\gcd(p, n) = p$.
            \end{proof}
        \2 \begin{proof}
                Let $g = gcd(p, n)$. Suppose that $p$ is prime, and seeking a contradiction, for all $n \in \mathbb{Z}$:
                \begin{equation}
                    \begin{aligned}
                        g &\neq 1 \\
                        g &\neq p                        
                    \end{aligned}
                    \label{eqn:primecontradictassumption}
                \end{equation}

                By definition of gcd, it is known that $g | p$, and by definition of divides, for some $i \in \mathbb{Z}$:
                \begin{equation}
                    p = ig
                    \label{eqn:pig2}
                \end{equation}

                $i$ and $g$ are therefore factors of $p$, and by definition of prime, $i, g \in \{1, p\}$.

                \textbf{Case 1.} Suppose that $i = 1$. Then by (\ref{eqn:pig2}), $g = p$. \contradiction

                This is a contradiction because we assumed in (\ref{eqn:primecontradictassumption}) that $g \neq p$.

                \textbf{Case 2.} Suppose that $i = p$. Then by (\ref{eqn:pig2}), $g = 1$. \contradiction

                This is a contradiction because we assumed in (\ref{eqn:primecontradictassumption}) that $g \neq 1$.

                Since all cases contain contradictions, the original assumption contains a contradiction. 
                Thus, by proof by contradiction, if $p$ is prime, then $\gcd(p, n) = 1$ or $\gcd(p, n) = p$.
            \end{proof}

    \1 \begin{theorem}
        Let $A$ be a set. $\emptyset \times A = \emptyset$.
        \label{thm:cartesianemptyset}
    \end{theorem}
    \begin{proof}
        \textbf{Base case.} Consider $A = \emptyset$ (a set of 0 elements). 
        
        $\emptyset \times A = \emptyset \times \emptyset = \{\} \times \{\}$, which by inspection, equals $\emptyset$.
        Therefore theorem \ref{thm:cartesianemptyset} holds true for $A = \emptyset$.

        \textbf{Inductive Hypothesis.} Let $B$, a set of $n$ elements, be a strict subset of the universe $U$.
        Suppose that $\emptyset \times B = \emptyset$.

        \textbf{Inductive Step.} Let $k \in U \setminus B$, and $A = B \cup \{k\}$, a set of $n+1$ elements. 
        Consider the following:
        \begin{equation}
            \emptyset \times ( B \cup \{ k\} ) 
        \end{equation}

        Applying the distributive property for sets, this is equal to:
        \begin{equation}
            (\emptyset \times B) \cup (\emptyset \times \{ k \} )
        \end{equation}
    
        By inspection, $\emptyset \times \{ k \} = \{\} \times \{ k \} = \emptyset$
        \begin{equation}
            (\emptyset \times B) \cup \emptyset
        \end{equation}

        By the inductive hypothesis, this equals:
        \begin{equation}
            \emptyset \cup \emptyset = \emptyset  
        \end{equation}

        Therefore, by proof by mathematical induction, $\emptyset \times A = \emptyset$
        for all sets $A$.
    \end{proof}

    \1 The inductive hypothesis is loosely defined. It supposes that all the horses in some set 
    have the same color, but does not specify which set exactly. Additionally, it is improperly 
    applied due to this loose definition. 
    \1 
        \2 $1^2 = \frac{1(1 + 1)(2 + 1)}{6}$
        \2 Left-hand side:
        \begin{equation}
            1^2 = 1
        \end{equation}

        Right-hand side:
        \begin{equation}
            \begin{aligned}
                \frac{1(1+1)(2 + 1)}{6} &= \frac{1(2)(3)}{6}  \\
                &= \frac{6}{6}  \\
                &= 1
            \end{aligned}
        \end{equation}
        \2 Suppose that the following is true:
        \begin{equation}
            \sum\limits^{k}_{i=1} i^2 = \frac{k(k + 1)(2k + 1)}{6}
        \end{equation}
        \2 Consider the case when $n = k + 1$:
        \begin{equation}
            \begin{aligned}
                \sum\limits^{k+1}_{i=1} i^2 &= (k+1)^2 + \sum\limits^{k}_{i=1} i^2  \\
            \end{aligned}
        \end{equation}

        By the inductive hypothesis, this equals
        \begin{equation}
            \begin{aligned}
                &(k+1)^2 + \frac{k(k + 1)(2k + 1)}{6}  \\
                =\ & \frac{6(k+1)^2 + k(k + 1)(2k + 1)}{6} \\
                =\ & \frac{(k+1)\left[6(k+1) + k(2k+1)\right]}{6}  \\
                =\ & \frac{(k+1)\left(6k + 6 + 2k^2 + k\right)}{6} \\
                =\ & \frac{(k+1)\left(2k^2 + 7k + 6\right)}{6}  \\
                =\ & \frac{(k+1)(k + 2)(2k+3)}{6}  \\
                =\ & \frac{(k+1)\left[(k + 1)+1\right]\left[2(k+1)+1\right]}{6}  \\
            \end{aligned}
        \end{equation}

        Therefore, by proof by mathematical induction,
        \begin{equation}
            \sum\limits^{n}_{i=1} i^2 = \frac{n(n + 1)(2n + 1)}{6}
        \end{equation} 
    
    \1 \begin{theorem}
        For all $n \in \mathbb{Z}^+$,
        \begin{equation}
            \sum\limits^{n}_{i=1} (i\cdot i!) = (n + 1)! - 1
            \label{eqn:thmfactorial}
        \end{equation}
    \end{theorem}

    \begin{proof}
        \textbf{Base case.} Consider when $n = 1$.

        \begin{equation}
            \begin{aligned}
                1\cdot 1! &= (1 + 1)! - 1 \\
                1 * 1 &= 2! - 1 \\
                1 &= 1
            \end{aligned}
        \end{equation}

        Therefore, the theorem holds for $n = 1$.

        \textbf{Inductive Hypothesis.} Suppose the following to be true for some $k \in \mathbb{N}$:
        \begin{equation}
            \sum\limits^{k}_{i=1} (i\cdot i!) = (k + 1)! - 1
        \end{equation}

        \textbf{Inductive Step.} Consider the case when $n = k + 1$.
        \begin{equation}
            \begin{aligned}
                &\sum\limits^{k+1}_{i=1} (i\cdot i!) \\
                = (k+1)(k+1)! + &\sum\limits^{k}_{i=1} (i\cdot i!)
            \end{aligned}
        \end{equation}

        By the inductive hypothesis, this is equal to:
        \begin{equation}
            \begin{aligned}
                &\ (k+1)(k+1)! + (k + 1)! - 1 \\ 
                =&\ (k + 1 + 1)(k + 1)! - 1  \\
                =&\ (k + 2)(k + 1)! - 1  \\
                =&\ (k + 2)! - 1  \\
                =&\ [(k + 1) + 1]! - 1  \\
            \end{aligned}
        \end{equation}

        Therefore, by proof by mathematical induction, (\ref{eqn:thmfactorial}) is true.

    \end{proof}

    \1 \begin{theorem}
        For all $n \in \mathbb{Z}^+$, $3\ |\ (n^3+2n)$.
        \label{thm:3divs}
    \end{theorem}

    \begin{proof}
        \textbf{Base case.} Consider when $n = 1$.
        \begin{equation}
            \begin{aligned}
                3\ |\ (1^3 + 2*1) &\equiv 3\ |\ (1 + 2)  \\
                &\equiv 3 \ | \ 3 \\
                &\equiv T
            \end{aligned}
        \end{equation}
        The theorem holds for $n = 1$.

        \textbf{Inductive Hypothesis.} Suppose the following is true for some $k \in \mathbb{Z}^+$:
        \begin{equation}
            3 \ | \ (k^3 + 2k)
        \end{equation}

        By definition of divides, this implies that there exists some $i \in \mathbb{Z}$ such that:
        \begin{equation}
            3i = k^3 + 2k
        \end{equation}

        \textbf{Inductive Step.} Let $n = k + 1$. Consider the following:
        \begin{equation}
            \begin{aligned}
                (k + 1)^3 + 2(k + 1) &= k^3 + 3k^2 + 3k + 1 + 2k + 2 \\
                &= (k^3 + 2k) + (3k^2 + 3k + 3) \\
                &= (k^3 + 2k) + 3(k^2 + k + 1) \\
            \end{aligned}
        \end{equation}

        By the inductive hypothesis, this is equivalent to:
        \begin{equation}
            \begin{aligned}
                &3i + 3(k^2 + k + 1) \\
                =\ &3(i + k^2 + k + 3) 
            \end{aligned}
        \end{equation}

        Let $m = i + k^2 + k + 3$. Thus:
        \begin{equation}
            (k + 1)^3 + 2(k + 1) = 3m
        \end{equation}

        By definition of divides:
        \begin{equation}
            3 \ | \left[(k + 1)^3 + 2(k + 1)\right]
        \end{equation}

        Therefore, by proof by mathematical induction, $3\ |\ (n^3+2n)$ for all positive integers $n$.

    \end{proof}

    \1 
    \2 \begin{theorem}
        Let $A, B_1, B_2, \dots, B_n$ be sets. The following is true:
        \begin{equation}
            A \cup \bigcap\limits^{n}_{i=1} B_i = \bigcap\limits^n_{i=1} (A \cup B_i)               
        \end{equation}
    \end{theorem}

    \begin{proof}
        \textbf{Base case.} Consider the case when $n = 1$.
        \begin{equation}
            \begin{aligned}
                A \cup \bigcap\limits^{1}_{i=1} B_i &= A \cup B_i = \bigcap\limits^{1}_{i=1} (A \cup B_i)
            \end{aligned}
        \end{equation}

        Thus, the theorem holds for $n = 1$.

        \textbf{Inductive Hypothesis.} Suppose the following is true:
        \begin{equation}
            A \cup \bigcap\limits^{k}_{i=1} B_i = \bigcap\limits^k_{i=1} (A \cup B_i)               
        \end{equation}

        \textbf{Inductive Step.} Consider the case when $n = k + 1$:
        \begin{equation}
            A\cup \bigcap\limits^{k + 1}_{i=1} B_i
            = A \cup \left( B_{k+1} \cap \bigcap\limits^{k}_{i=1} B_i \right) 
        \end{equation}

        By the distributive property of sets,
        \begin{equation}
            \left(A \cup B_{k+1}\right) \cap \left(A \cup \bigcap\limits^{k}_{i=1} B_i \right)
        \end{equation}

        By the inductive hypothesis, this is equal to
        \begin{equation}
            \begin{aligned}
                \left(A \cup B_{k+1}\right) \cap \bigcap\limits^k_{i=1} (A \cup B_i) = \bigcap\limits^{k+1}_{i=1} (A \cup B_i) 
            \end{aligned}
        \end{equation}

        Thus, by the principle of mathematical induction, 
        \begin{equation}
            A \cup \bigcap\limits^{n}_{i=1} B_i = \bigcap\limits^n_{i=1} (A \cup B_i)               
        \end{equation}

    \end{proof}

    \2 \begin{theorem}
        Let $A, B_1, B_2, \dots, B_n$ be sets. The following is true:
        \begin{equation}
            A \cap \bigcup\limits^{n}_{i=1} B_i = \bigcup\limits^n_{i=1} (A \cap B_i)               
        \end{equation}
    \end{theorem}

    \begin{proof}
        \textbf{Base case.} Consider the case when $n = 1$.
        \begin{equation}
            \begin{aligned}
                A \cap \bigcup\limits^{1}_{i=1} B_i &= A \cap B_i = \bigcup\limits^{1}_{i=1} (A \cap B_i)
            \end{aligned}
        \end{equation}

        Thus, the theorem holds for $n = 1$.

        \textbf{Inductive Hypothesis.} Suppose the following is true:
        \begin{equation}
            A \cup \bigcap\limits^{k}_{i=1} B_i = \bigcap\limits^k_{i=1} (A \cup B_i)
        \end{equation}

        \textbf{Inductive Step.} Consider the case when $n = k + 1$:
        \begin{equation}
            A\cap \bigcup\limits^{k + 1}_{i=1} B_i
            = A \cap \left( B_{k+1} \cup \bigcup\limits^{k}_{i=1} B_i \right) 
        \end{equation}

        By the distributive property of sets,
        \begin{equation}
            \left(A \cap B_{k+1}\right) \cup \left(A \cap \bigcup\limits^{k}_{i=1} B_i \right)
        \end{equation}

        By the inductive hypothesis, this is equal to
        \begin{equation}
            \begin{aligned}
                \left(A \cap B_{k+1}\right) \cup \bigcup\limits^k_{i=1} (A \cap B_i) = \bigcup\limits^{k+1}_{i=1} (A \cap B_i) 
            \end{aligned}
        \end{equation}

        Thus, by the principle of mathematical induction, 
        \begin{equation}
            A \cap \bigcup\limits^{n}_{i=1} B_i = \bigcup\limits^n_{i=1} (A \cap B_i)               
        \end{equation}

    \end{proof}

\end{outline} 


\newgeometry{margin=2cm}
\begin{landscape}
    \section{Scratch Work}

    \centering
    \thispagestyle{empty}
    \includegraphics[width=0.95\linewidth]{sw1.jpg}

    \pagebreak
    
    \thispagestyle{empty}
    \includegraphics[width=0.95\linewidth]{sw2.jpg}
\end{landscape}

\restoregeometry
    
\end{document}
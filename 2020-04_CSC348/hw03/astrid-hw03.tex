\documentclass{article}
\usepackage{outlines}
\usepackage{hyperref}
\usepackage{amssymb}
\usepackage{amsfonts}
\usepackage{amsmath}
\usepackage{amsthm}
\usepackage{blindtext}

\author{Astrid Augusta Yu}
\title{CSC 348 -- Homework \#2}
\date{\today}

\renewcommand\qedsymbol{\texttt{d(\textbf{\^{}\_\^{}})>}}
\newtheorem{theorem}{Theorem}
\newtheorem{lemma}{Lemma}
\newtheorem{case}{Case}
\newtheorem{subcase}{Case}
\numberwithin{subcase}{case}

\begin{document}
\maketitle
\begin{outline}[enumerate]

    \1 \begin{theorem}
        Let $A$ and $B$ be sets with some universe $U$. $A \setminus B = A \cap \overline{B}$.

        \begin{proof}
            $(\rightarrow)$ Suppose $x \in (A \setminus B)$. By definition of set difference, 
            \begin{equation}
                \begin{aligned}
                    x \in A \wedge x \notin B
                \end{aligned}
            \end{equation}
            
            By definition of complement:
            \begin{equation}
                \begin{aligned}
                    x \in A \wedge x \in \overline{B}
                \end{aligned}
            \end{equation}

            By definition of intersection:
            \begin{equation}
                \begin{aligned}
                    x \in (A \setminus \overline{B}) 
                \end{aligned}
            \end{equation}

            Therefore, $x \in (A \setminus B) \rightarrow x \in (A \cap \overline{B})$.

            By definition of subset, $(A \setminus B) \subseteq (A \cap \overline{B})$.

            $(\leftarrow)$ Let $x \in U$. Suppose $x \in A \cap \overline{B}$. 
            
            By a symmetric argument, $x \in (A \cap \overline{B}) \rightarrow x \in (A \setminus B)$.  

            By definition of subset, $(A \cap \overline{B}) \subseteq (A \setminus B)$.

            Because $(A \setminus B) \subseteq (A \cap \overline{B})$ and $(A \cap \overline{B}) \subseteq (A \setminus B)$, 
            by definition of equivalent sets, $(A \setminus B) = (A \cap \overline{B})$.
        \end{proof} 
    \end{theorem}

    \1 \begin{theorem}
        Let A, B, and C be sets. $A \cap (B \cup C) = (A \cap B) \cup (A \cap C)$.
        \begin{proof}
            $(\rightarrow)$ Let $x \in A \cap (B \cup C)$. By definition of intersection: 
            \begin{equation}
                (x \in A) \wedge [x \in (B \cup C)]
            \end{equation}

            By definition of union: 
            \begin{equation}
                (x \in A) \wedge (x \in B \vee x \in C)
            \end{equation}

            Using the boolean distributive property: 
            \begin{equation}
                (x \in A \wedge x \in B) \vee (x \in A \wedge x \in C)
            \end{equation}

            By definition of intersection: 
            \begin{equation}
                (x \in A \cap B) \vee (x \in A \cap B)
            \end{equation}

            By definition of union:
            \begin{equation}
                x \in (A \cap B) \cup (A \cap B)
            \end{equation}

            Therefore, $x \in [A \cap (B \cup C)] \rightarrow x \in [(A \cap B) \cup (A \cap C)]$.

            By definition of subset, $A \cap (B \cup C) \subseteq (A \cap B) \cup (A \cap C)$.

            $(\leftarrow)$ Suppose $x \in (A \cap B) \cup (A \cap C)$. 
            
            By a symmetric argument,
            \begin{equation}
                x \in [(A \cap B) \cup (A \cap C)] \rightarrow x \in [A \cap (B \cup C)]  
            \end{equation}
            
            By definition of subset, 
            \begin{equation}
                (A \cap B) \cup (A \cap C) \subseteq A \cap (B \cup C)
            \end{equation}

            Therefore, by definition of equivalent sets, 
            \begin{equation}
                A \cap (B \cup C) = (A \cap B) \cup (A \cap C)
            \end{equation}
        \end{proof}
        
    \end{theorem}

    \1 \begin{theorem}
        Let A, B, and C be sets. $A \cup (B \cap C) = (A \cup B) \cap (A \cup C)$.

        \begin{proof}
            $(\rightarrow)$ Let $x \in A \cup (B \cap C)$. By definition of union: 
            \begin{equation}
                (x \in A) \vee [x \in (B \cap C)]
            \end{equation}

            By definition of intersection: 
            \begin{equation}
                (x \in A) \vee (x \in B \wedge x \in C)
            \end{equation}
        
            Using the boolean distributive property: 
            \begin{equation}
                (x \in A \vee x \in B) \wedge (x \in A \vee x \in C)
            \end{equation}

            By definition of union: 
            \begin{equation}
                (x \in A \cup B) \wedge (x \in A \cup B)
            \end{equation}

            By definition of intersection:
            \begin{equation}
                x \in (A \cup B) \cap (A \cup B)
            \end{equation}

            Therefore, $x \in [A \cup (B \cap C)] \rightarrow x \in [(A \cup B) \cap (A \cup C)]$.

            By definition of subset, $A \cup (B \cap C) \subseteq (A \cup B) \cap (A \cup C)$.

            $(\leftarrow)$ Suppose $x \in (A \cup B) \cap (A \cup C)$. 
            
            By a symmetric argument,
            \begin{equation}
                x \in [(A \cup B) \cap (A \cup C)] \rightarrow x \in [A \cup (B \cap C)]  
            \end{equation}
            
            By definition of subset, 
            \begin{equation}
                (A \cup B) \cap (A \cup C) \subseteq A \cup (B \cap C)
            \end{equation}

            Therefore, by definition of equivalent sets, 
            \begin{equation}
                A \cup (B \cap C) = (A \cup B) \cap (A \cup C)
            \end{equation}
        \end{proof}
    \end{theorem}

    \1 \begin{theorem}
            Let $A$ and $B$ be sets. $\overline{A \cup B} = \overline{A} \cap \overline{B}$.
        \end{theorem}
        \begin{proof}
            $(\rightarrow)$ Suppose $x \in \overline{A \cup B}$. By 
            definition of set complement, 
            \begin{equation}
                \overline{x \in (A \cup B)}
            \end{equation} 

            By definition of union, 
            \begin{equation}
                \overline{(x \in A) \vee (x \in B)}
            \end{equation}

            Using DeMorgan's Law, this is equivalent to 
            \begin{equation}
                \overline{x \in A} \wedge \overline{x \in B}
            \end{equation}

            By definition of intersection,
            \begin{equation}
                x \in (\overline{A} \cap \overline{B})
            \end{equation}
            
            Therefore, $x \in \overline{A \cup B} \rightarrow x \in \overline{A} \cap \overline{B}$.

            $(\leftarrow)$ Suppose $x \in \overline{A} \cap \overline{B}$. 
            
            By a symmetric argument, 
            \begin{equation}
                x \in \overline{A} \cap \overline{B} \rightarrow x \in \overline{A \cup B}  
            \end{equation}
            
            By definition of subset, the following are true:
            \begin{equation}
                \begin{aligned}
                    \overline{A} \cap \overline{B} \subseteq \overline{A \cup B}   \\
                    \overline{A \cup B} \subseteq \overline{A} \cap \overline{B}  
                \end{aligned}
            \end{equation}

            Therefore, by definition of set equivalence,
            \begin{equation}
                \overline{A \cup B} = \overline{A} \cap \overline{B}
            \end{equation}
        \end{proof}
    
    \1 \begin{theorem}
        Let $A$ and $B$ be sets. $\overline{A \cup B} = \overline{A} \cap \overline{B}$.
        \end{theorem}
        \begin{proof}
            $(\rightarrow)$ Suppose $x \in \overline{A \cap B}$. By 
            definition of set complement, 
            \begin{equation}
                \overline{x \in (A \cap B)}
            \end{equation} 

            By definition of intersection, 
            \begin{equation}
                \overline{(x \in A) \wedge (x \in B)}
            \end{equation}

            Using DeMorgan's Law, this is equivalent to 
            \begin{equation}
                \overline{x \in A} \vee \overline{x \in B}
            \end{equation}

            By definition of union,
            \begin{equation}
                x \in (\overline{A} \cup \overline{B})
            \end{equation}
            
            Therefore, $x \in \overline{A \cap B} \rightarrow x \in \overline{A} \cup \overline{B}$.

            $(\leftarrow)$ Suppose $x \in \overline{A} \cup \overline{B}$. 
            
            By a symmetric argument, 
            \begin{equation}
                x \in \overline{A} \cup \overline{B} \rightarrow x \in \overline{A \cap B}  
            \end{equation}
            
            By definition of subset, the following are true:
            \begin{equation}
                \begin{aligned}
                    \overline{A} \cup \overline{B} \subseteq \overline{A \cap B}   \\
                    \overline{A \cap B} \subseteq \overline{A} \cup \overline{B}  
                \end{aligned}
            \end{equation}

            Therefore, by definition of set equivalence,
            \begin{equation}
                \overline{A \cap B} = \overline{A} \cup \overline{B}
            \end{equation}
        \end{proof}

    \1 \begin{theorem}
            \label{thm:q6}
            Let $m, n, p \in \mathbb{Z}$. If $m + n$ is even and $n + p$ is even, then 
            $m + p$ is even.
        \end{theorem}
            
        First, let us additionally define the following lemma:

        \begin{lemma}
            \label{thm:evenlemma}
            Let $a, b \in \mathbb{Z}$. $a + b$ is even if and only if $a$ and $b$ have 
            the same parity, and odd if and only if $a$ and $b$ have different parities.
        \end{lemma}
            
        \begin{proof}[Proof of Lemma \ref{thm:evenlemma}]
            
            \textbf{Case 1.} $a$ and $b$ are both even.

            By definition of even:
            \begin{equation}
                \begin{aligned}
                    a &= 2k \\ 
                    b &= 2l
                \end{aligned}
            \end{equation}

            Therefore,
            \begin{equation}
                \begin{aligned}
                    a + b &= 2k + 2l \\
                    &= 2(k + l) 
                \end{aligned}
            \end{equation}

            Let $m = k + l$. Then
            \begin{equation}
                a + b = 2m
            \end{equation}

            By definition of even, $a + b$ is even.  
           
            \textbf{Case 2.} $a$ and $b$ are both odd.

            By definition of odd:
            \begin{equation}
                \begin{aligned}
                    a &= 2k + 1 \\
                    b &= 2l + 1
                \end{aligned}
            \end{equation}

            Therefore, 
            \begin{equation}
                \begin{aligned}
                    a + b &= 2k + 1 + 2l + 1 \\ 
                    &= 2(k + l + 1) 
                \end{aligned}
            \end{equation}
/lnoy
            Let $m = k + l + 1$. Then,
            \begin{equation}
                a + b = 2m
            \end{equation}

            By definition of even, a + b is even.

            \textbf{Case 3.} $a$ and $b$ have different parity. WLOG let $a$ be even and $b$ be odd.

            By definition of even:
            \begin{equation}
                a = 2k
            \end{equation}

            By definition of odd:
            \begin{equation}
                b = 2l + 1
            \end{equation}

            Therefore,
            \begin{equation}
                \begin{aligned}
                    a + b &= 2k + 2l + 1 \\
                    &= 2(k + l) + 1
                \end{aligned}
            \end{equation}

            Let $m = k + l$. Then,
            \begin{equation}
                a + b = 2m + 1
            \end{equation}

            By definition of odd, a + b is odd.

            \textbf{In Summary:}

            \begin{center}
                \begin{tabular}{c c | c}
                    $a$ parity & $b$ parity & $a + b$ parity \\
                    \hline
                    even & even & even \\ 
                    even & odd & odd \\ 
                    odd & even & odd \\ 
                    odd & odd & even
                \end{tabular}
            \end{center}

            Therefore, $a + b$ is only even when a's parity $=$ b's parity, and only odd when 
            a's parity $\neq$ b's parity, concluding the lemma.
        \end{proof}

        \begin{proof}[Proof of Theorem \ref{thm:q6}]
            Let $m, n, p \in \mathbb{Z}$. Suppose $m + n$ is even and $n + p$ is even. 
            
            WLOG, let $P \in \{even, odd\}$ be the parity of $m$.

            By Lemma \ref{thm:evenlemma}, because $m + n$ is even and $m$ has parity $P$, then $n$
            must have parity $P$ as well. 

            By the same lemma, because $n + p$ is even and $n$ has parity $P$, then $p$ must  have 
            parity $P$ as well.

            Also by the same lemma, because $m$ and $p$ have the same parity $P$, $m + p$ must be even.  

            Therefore, if $m + n$ is even and $n + p$ is even, then $m + p$ is even.
        \end{proof}

    \1 \begin{theorem}
        \begin{equation}
            \min(x, y) = \frac{x + y - | x - y |}{2}
        \end{equation}
    \end{theorem}

    \begin{proof}
        By definition of absolute value:
        \begin{equation}
            |x - y| = \begin{cases}
                x - y &if x \geq y \\
                y - x &if y > x
            \end{cases}
        \end{equation}

        In both cases, it is always equal to the positive difference, no matter if x is greater or y is greater.

        WLOG suppose $x \geq y$. Thus, $|x - y| = x - y$. Substituting into the equation:
        \begin{equation}
            \begin{aligned}
                \min(x, y) &= \frac{x + y - (x - y)}{2} \\
                &= \frac{x + y - x + y}{2} \\
                &= \frac{2y}{2} \\
                \min(x, y) &= y \\
            \end{aligned}
        \end{equation}

        This statement is true by definition of $\min$. 
        
        Therefore, $\min(x, y) = \frac{x + y - | x - y |}{2}$
    \end{proof}

    \1 \begin{theorem}
        \begin{equation}
            max(x, y) = \frac{x + y - | x - y |}{2}
        \end{equation}
    \end{theorem}

    \begin{proof}
        WLOG suppose $x \geq y$. Therefore, $|x - y| = x - y$. Substituting into the equation:
        \begin{equation}
            \begin{aligned}
                \max(x, y) = x &= \frac{x + y + (x - y)}{2} \\
                &= \frac{2x}{2} \\
                \max(x, y) &= x \\
            \end{aligned}
        \end{equation}

        This statement is true by definition of $\max$. 
        
        Therefore, $\max(x, y) = \frac{x + y + | x - y |}{2}$
    \end{proof}

    \1 
        \2 True. $k = 9$.
        \2 True. $k = -4$.
        \2 False. $4 > 1$ so there can't be a $k \in \mathbb{Z}$ such that $4k = 1$.
        \2 False. 7 is prime, and $2 \notin \{1, 7\}$.
        \2 False. $0k = 17$ means that $0 = 17$ which means it is false for all $k$.
        \2 False. 17 is prime, and $3 \notin \{1, 17\}$.
        \2 True. $k = 0$.

    \1
        \2 False 
        \2 True 
        \2 True
        \2 True. Let $x \in \mathbb{Z}$ and $k = 0$. Therefore $xk = 0$. By definition of 
            divides, $\forall x (x | 0)$.
        \2 True. Let $y \in \mathbb{Z}$, and $k = y$. $1k = y$. By definition of divides, 
            $\forall y(1 | y)$.
        \2 False. By way of contradiction, assume $\forall x \forall y (x | y)$ to be true. 
            Let $x = 10$ and $y = 6$. Therefore, according to the statement, 
            $(x | y) \equiv (10 | 6)$. However, this is false, contradicting
            the original premise that $\forall x \forall y (x | y)$. Therefore, 
            the statement is false.
        \2 True, for $x = 1$ as proven above.

    \1 \begin{theorem}
        Let $a, b, c \in \mathbb{Z}$. If $a | b$ and $b | c$ then $a | c$.
    \end{theorem}

    \begin{proof}
        By definition of divides, the following are true:
        \begin{itemize}
            \item $ak = b$ for some $k \in \mathbb{Z}$
            \item $bl = c$ for some $l \in \mathbb{Z}$
        \end{itemize}

        Therefore, we can multiply the equations together:
        \begin{equation}
            \begin{aligned}
                (ak)(bl) &= (b)(c) \\
                aklb &= cb 
            \end{aligned}
        \end{equation}

        Then, we can unmultiply\footnote{Because division is not closed on 
        $\mathbb{Z}$ :(} $b$ from both sides: 
        \begin{equation}
            akl = c
        \end{equation}

        Let $m = kl$. Then, $am = c$. Therefore, by definition of divides, 
        $a | c$.
    \end{proof}

\end{outline} 
\end{document}
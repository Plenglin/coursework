\documentclass[12pt]{article}

\usepackage{verbatim}
\usepackage{graphicx, float}
\usepackage{enumerate}
\usepackage{amsmath}
\usepackage{bookmark}
\usepackage{geometry}[margin=1in]
\usepackage{hyperref}


\author{Astrid Yu}
\title{Lab 5 Prelab}
\DeclareMathOperator{\V}{V}
\DeclareMathOperator{\A}{A}
\DeclareMathOperator{\F}{F}
\DeclareMathOperator{\mA}{mA}
\DeclareMathOperator{\mW}{mW}
\DeclareMathOperator\bit{bit}
\DeclareMathOperator\byte{byte}
\DeclareMathOperator\word{word}

\begin{document}
\maketitle

\begin{enumerate}
    \item The selected resistances were $R1 = 39\Omega, R2=10\Omega$.
    
        The SPICE output is $V_{R2} = 1.0204\V, I_{R2} = 102.0\mA$.
        
        The delivered power is $V_{R2}\cdot I_{R2} = 104.1\mW$.
    
        \includegraphics[width=1\linewidth]{lab5pre1.png}

    \item \begin{equation}
        \begin{aligned}
            V &= \frac{R}{R + 2k\Omega}\cdot 5\V \\
            V \cdot R + V \cdot 2k\Omega &= R\cdot 5\V \\
            V \cdot 2k\Omega &= 5\V \cdot R - V \cdot R \\
            V \cdot 2k\Omega &= (5\V - V) \cdot R \\
            \frac{V \cdot 2k\Omega}{5\V - V} &= R
        \end{aligned}
    \end{equation}
    \item \begin{equation}
        \begin{aligned}
            V = \frac{5\V}{2^{10}}\cdot X
        \end{aligned}
    \end{equation}
    \item Let $T$ be the ambient temperature and $R$ be the resistance of the thermistor. Their relationship was found 
        to be
        \begin{equation}
            T=(0.1056\frac{^\circ \F}{\Omega}) \cdot R -146.4 ^\circ\F
        \end{equation}

        \includegraphics[width=\linewidth]{prelab-chart.png}
\end{enumerate}

\end{document}

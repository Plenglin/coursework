\documentclass[12pt]{article}

\usepackage{verbatim}
\usepackage{graphicx, float}
\usepackage{enumerate}
\usepackage{amsmath}
\usepackage{bookmark}
\usepackage{pdflscape}
\usepackage{multirow}
\usepackage{hyperref}

\DeclareMathOperator{\V}{V}
\DeclareMathOperator{\A}{A}
\DeclareMathOperator{\F}{F}
\DeclareMathOperator{\mA}{mA}
\DeclareMathOperator{\mW}{mW}
\DeclareMathOperator\bit{bit}
\DeclareMathOperator\byte{byte}
\DeclareMathOperator\word{word}

\author{Astrid Yu}
\title{Exp \#7 Prelab}

\begin{document}
\maketitle

\begin{enumerate}
    \item $V_p = V_1$ because the op-amp is in a feedback loop. 
    
    \item The current flowing through the $V_1$ node is given by the following expression:
    $$\frac{-V_1}{820\Omega} + 0A + I_{load} = 0$$
    Therefore, 
    $$I_{load} = \frac{V_1}{820\Omega}$$
    
    \item The sensor voltage $V_s$ is expressed by
    $$V_s = I_{load}\cdot R_{sensor} = V_1 \cdot \frac{R_{sensor}}{820\Omega}$$
    
    \item $$A_0 = V_1 \cdot \frac{R_{sensor}}{820\Omega} + V_1 = V_1 \cdot \left(1 + \frac{R_{sensor}}{820\Omega}\right)$$
    $$R_{sensor} = 820\Omega\cdot\left(\frac{A_0}{V_1} - 1\right)$$

\end{enumerate}

\end{document}

\chapter{Simple Harmonic Motion}

Book chapter 17

\section{2020-09-14}

\subsection{Syllabus}

\begin{itemize}
    \item \href{https://calpoly.zoom.us/j/680327688}{Zoom meeting}
    \item Yay no textbook
\end{itemize}

\subsection{Lecture}

\subsubsection{Basics of Oscillations}

\begin{itemize}
    \item Oscillation = motion that repeats itself, back and forth around equilibrium position. Most important is Simple Harmonic Motion (SHM), sinusoidal
    \item Spring goes from expansion to contraction to expansion, etc. Restoring force occurs when stretched or compressed, goes against spring's displacement
    \item If graph repeats, it's oscillatory motion. If position = sinusoidal, then it's SHM. 
\end{itemize}

\begin{figure}
    \centering
    \includesvg{week1/spring.svg}
    \caption{Spring}
\end{figure}

\begin{figure}
    \centering
    
    $$x(t) = Acos(\omega t + \phi_0)$$
    $$\omega = 2 \pi f$$
    
    \caption{Equations for SHM}
\end{figure}

\begin{itemize}
    \item Amplitude, period, and time let you derive position
\end{itemize}

\subsection{Lab Partners}

\begin{itemize}
    \item Drew (EE)
    \item Kishor (ME)
    \item Astrid (CS)
\end{itemize}

\section{2020-09-16}

\begin{itemize}
    \item Object always goes around equilibrium point
    \item Linear restoring force
\end{itemize}

2 main prototypes of SHM:
\begin{outline}
    \1 Mass on a spring without any friction. (With friction = damped oscillation)
    \1 Pendulum
        \2 Generally not SHM, but for small displacement angles, it is approximately linear and SHM
\end{outline}

$$\vv{F}_{spring} = -k\vv{x}$$

where $\vv{x} = $ displacement, $\vv{F}_{spring} = $ spring force, and $\vv{k} = $ spring constant.

\textbf{$\vv{F}_{spring}$ is a linear force.}

Motion equations:

$$x(t) = A\cos(\omega t)$$

$$v(t) = \frac{d}{dt}x(t) = -\omega A \sin(\omega t)$$

$$v_{max} = \omega A$$

$\si{\radian}$ is not a physical unit. So, $\frac{\si{\radian}}{\si{\sec}} \cdot \si{\metre} = \frac{\si{\metre}}{\si{\sec}}$

\begin{figure}
    \centering
    \includesvg{week1/shm-path.svg}
    \caption{Position and velocity of a object experiencing SHM. They are $90^\circ$ out of phase. $|v| = v_{max}$ when $x = 0$, $v = 0$ when $|x| = A$}
\end{figure}

\subsubsection{Question}

Mass on spring $20\si{\cm}$ right, released $t = 0\sec$, $15$ oscillations in $10\sec$

\begin{enumerate}[a.]
    \item $$T = \frac{10\sec}{15} = \frac{2}{3}\sec$$
    \item $$\omega = \frac{2\pi}{T} = 3\pi$$
    $$v_{max} = \omega \cdot A = 0.6\pi \si{\m\per\sec}$$
    \item $$v(.8) = -1.78 \si{\cm\per\sec}$$
    $$x(.8) = .0618\si{\m}$$
\end{enumerate}

\subsubsection{Pendulum}

$$T = 2\pi \sqrt{\frac{L}{g}}$$

$$T^2 = \left(\frac{4\pi^2}{g}\right)L$$

$$T^2 \propto L$$

\section{2020-09-18}

\subsection{Phase}

Phases describe repeated configurations in a cycle.

\subsubsection{General SHM equations}

$$x(t) = A\cos(\omega t + \phi_0)$$
$$v(t) = -\omega A\sin(\omega t + \phi_0)$$
$$a(t) = -\omega^2 A\cos(\omega t + \phi_0)$$

where $\phi_0$ is the phase constant in radians

Trivial case is when $\phi_0 = 0$

\subsection{Energy}

\subsubsection{Energy of a spring}
$$U_s(t) = \frac{1}{2}kx(t) = \frac{1}{2}k A \cos^2(\omega t + \phi_0)$$
$$E_k(t) = \frac{1}{2}mv(t) = \frac{1}{2}m \omega^2 A^2 \sin^2(\omega t + \phi_0)$$
$$E_{total} = \frac{1}{2} kA = \frac{1}{2} m \omega^2 A^2$$
$$\frac{k}{m} = \omega^2 A^2 = v_{max}^2$$

\subsection{Lab}

The following equation describes the relation between $T^2$ and $L$.

$$T^2 = \left(\frac{4\pi^2}{g}\right)L$$

From the linear regression, the following value was found:

$$\left(\frac{4\pi^2}{g}\right) = 4.04 \si{\sec^2\per\m}$$

Rearranging to solve for $g$,
$$g = \frac{4\pi}{4.04} = 9.77 \si{\m\per\s^2}$$